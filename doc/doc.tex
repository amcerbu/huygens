\documentclass{amsart}

\renewcommand{\labelenumi}{\bf\alph{enumi}.}
\renewcommand{\labelenumii}{\bf\alph{enumii}.}
\renewcommand{\labelenumiii}{\bf\roman{enumiii}.}
\renewcommand{\thesubsubsection}{\arabic{section}.\arabic{subsection}\,\alph{subsubsection}}

\usepackage{amssymb,amsmath,amsthm,amsfonts,bbm,accents,mathtools,dsfont,color,float}
\usepackage{enumerate}
\usepackage[margin=0.75in]{geometry}
\usepackage{graphicx,booktabs}
\usepackage{setspace}
\usepackage{stmaryrd}
\usepackage{centernot}
\usepackage{mathdots}
\usepackage{pgf,tikz}
\usepackage{mathrsfs}
\usetikzlibrary{arrows.meta}
\usetikzlibrary{positioning}
\usepackage{amsthm}
\usepackage{listings}
\usepackage{textcomp}
\usepackage{cancel}
\usepackage{arydshln}
\usepackage[normalem]{ulem}
\usepackage{import}
\usepackage{parskip}

\usepackage{algorithm, algorithmic}

\usepackage{pgfplots} 
\pgfplotsset{compat=1.17}

\allowdisplaybreaks

\DeclareMathOperator{\vspan}{span}
\DeclareMathOperator{\tr}{tr}
\DeclareMathOperator{\diag}{diag}
\DeclareMathOperator{\img}{im}
\DeclareMathOperator{\Var}{Var}
\DeclareMathOperator{\Cov}{Cov}
\DeclareMathOperator*{\esssup}{ess\,sup}
\DeclareMathOperator*{\essinf}{ess\,inf}
\DeclareMathOperator{\supp}{supp}
\DeclareMathOperator*{\essran}{ess\,ran}
\DeclareMathOperator{\Id}{Id}
\DeclareMathOperator{\Col}{Col}
\DeclareMathOperator{\Null}{Null}
\DeclareMathOperator{\proj}{proj}
\DeclareMathOperator{\rank}{rank}
\DeclareMathOperator{\sgn}{sgn}
\DeclareMathOperator{\Mat}{Mat}
\DeclareMathOperator{\class}{class}
\DeclareMathOperator{\err}{err}
\DeclareMathOperator{\argmin}{arg\,min}

\DeclarePairedDelimiter{\abs}{\lvert}{\rvert}
\DeclarePairedDelimiter{\norm}{\lVert}{\rVert}
\DeclarePairedDelimiter{\group}{\langle}{\rangle}
\DeclarePairedDelimiter{\set}{\{}{\}}
\DeclarePairedDelimiter{\term}{(}{)}
\DeclarePairedDelimiter{\brak}{[}{]}
\DeclarePairedDelimiter{\ceil}{\lceil}{\rceil}
\DeclarePairedDelimiter{\floor}{\lfloor}{\rfloor}

\newcommand{\R}{\mathbf{R}}
\newcommand{\N}{\mathbf{N}}
\newcommand{\C}{\mathbf{C}}
\newcommand{\Z}{\mathbf{Z}}
\newcommand{\Q}{\mathbf{Q}}
\newcommand{\diff}{\mathrm{d}}
\newcommand{\E}{\mathbf{E}}
\newcommand{\T}{\mathbf{T}}
\newcommand{\Pp}{\mathbf{P}}
\newcommand{\Hyp}{\mathbf{H}}
\newcommand{\1}{\mathbf{1}}
\newcommand{\F}{\mathbf{F}}
\renewcommand{\le}{\leqslant}
\renewcommand{\ge}{\geqslant}
\renewcommand{\v}{\mathbf}

\begingroup
\makeatletter
\@for\theoremstyle:=definition,remark,plain\do{%
\expandafter\g@addto@macro\csname th@\theoremstyle\endcsname{%
\addtolength\thm@preskip\parskip
}%
}
\endgroup

\newtheorem{thm}{Theorem}[section]
\newtheorem{innercustomthm}{Theorem}
\newenvironment{numthm}[1]
{\renewcommand\theinnercustomthm{#1}\innercustomthm}
{\endinnercustomthm}
\newtheorem{innercustomlem}{Lemma}
\newenvironment{numlem}[1]
{\renewcommand\theinnercustomlem{#1}\innercustomlem}
{\endinnercustomlem}
\newtheorem*{thm*}{Theorem}
\newtheorem*{lemma}{Lemma}
\newtheorem{prop}{Proposition}
\newtheorem{obs}{Observation}
\newtheorem{rmk}{Remark}
\newtheorem{cor}{Corollary}
\newtheorem{question}{Question}
\newtheorem{defn}{Definition}
\newtheorem{prob}{Problem}
\newtheorem{examp}{Example}
\newtheorem{claim}{Claim}



\definecolor{deepblue}{rgb}{0,0,0.8}
\definecolor{deepred}{rgb}{0.6,0,0}
\definecolor{neonorange}{rgb}{0.949,0.521,0}
\definecolor{deepgreen}{rgb}{0,0.5,0}
\definecolor{gray}{rgb}{0.4,0.4,0.4}
\definecolor{lightgray}{rgb}{0.7,0.7,0.7}

\newcommand\pythonstyle{\lstset{
language=Python,
upquote=true,
basicstyle={\small\ttfamily},
keywordstyle=\color{deepblue},
emphstyle=[1]\color{deepred},    % Custom highlighting style
emphstyle=[2]\color{neonorange},    % Custom highlighting style
emph=[1]{self, __init__},          % Custom highlighting
emph=[2]{True, False, None},
stringstyle=\color{deepgreen},
numberstyle=\tiny\color{gray},
commentstyle=\color{gray},
frame=tb,                         % Any extra options here
aboveskip=3mm,
  belowskip=3mm,
  showstringspaces=false,
  columns=fullflexible,
  keepspaces=true
}}


\lstnewenvironment{python}[1][] {\pythonstyle\lstset{#1}} {}
\newcommand\pythonexternal[2][]{{\pythonstyle\lstinputlisting[#1]{#2}}}
\newcommand\pythoninline[1]{{\pythonstyle\lstinline!#1!}}

\DeclareUnicodeCharacter{2212}{-}

\begin{document}
\begin{center} {\large \textbf{HUYGENS}\\
an odd kind of sympathy} \end{center}

Huygens is a library of tools for manipulating and synthesizing sound. It depends on \texttt{PortAudio} to provide access to audio hardware and on \texttt{RtMidi} for processing MIDI messages.

\section*{Class Hierarchy}
\begin{enumerate}
  \item \texttt{Wave<T>}: a wavetable. Initialized by passing a lambda of a phase (by default in $[0,1)$). Supports different forms of interpolated lookup. 
  
  \item \texttt{Oscillator<T>}: stores a frequency and a phase. The \texttt{tick()} method updates the phase according to the frequency. Also supports phase- and frequency-modulation, with an adjustable stiffness coefficient for smoothing sudden changes in these parameters. 
  
  \item \texttt{Synth<T>}: inherits from \texttt{Oscillator<T>}; also stores a pointer to a \texttt{Wave<T>}. A \texttt{Synth<T>} object has an \texttt{operator()} function which returns the wavetable evaluated at the current phase. 
  
  \item \texttt{Filter<T>}: initialized with a list of feedback and feedforward coefficients, $\set{a_j}_{j = 0}^N, \set{b_j}_{j = 1}^N$. Given a signal $\set{x_n}_{n \in \Z}$, an object of the \texttt{Filter<T>} class produces the signal $\set{(Tx)_n}_{n \in \Z}$ satisfying the recurrence
  \[ (Tx)_n = \sum_{j=0}^N a_j x_{n-j} - \sum_{j=1}^N b_j (Tx)_{n-j}. \]
  Has an \texttt{operator(T sample)} function, which takes a new sample as input and returns a filtered sample. Needs a call to \texttt{tick()} at sample rate. 

  Also supports initialization by a list of poles and zeros. 
  
  \item \texttt{Buffer<T>}: circular buffer object; internally, stores an array of some length and an ``origin'' position which is advanced with \texttt{tick()}. Member functions \texttt{write} and \texttt{accum} modify the array at the origin; \texttt{operator(T position)} looks up the contents of the buffer \texttt{position} samples in the past. 
  
  \item \texttt{Sinusoids<T>}: a bank of sinusoids in a series over some fundamental, with harmonicity and decay parameters. Supports sample rate modulation of the decay, harmonicity, and fundamental frequency.
  
  \item \texttt{Particle}, \texttt{Spring}, \texttt{Gravity}: classes for discrete Newtonian physics, currently in one dimension; a \texttt{Particle} has mass, position and velocity, and may accumulate forces with calls to \texttt{pull(double force)}. Objects of the class are initialized with a drag coefficient (the user supplies a time in seconds for relaxation from velocity $1$). The \texttt{Spring} and \texttt{Gravity} classes couple two particles; in the case of the first, a call to \texttt{tick()} pulls on the particles with a force proportional to their distance; in that of the second, with a force proportional to the product of their masses and inversely proportional to their distance squared. 
  
  \item \texttt{Polyphon<T>}: an ensemble of particles whose positions set the frequencies of an additive synthesizer. A ``note'' is a series of particles (whose positions represent pitches, not frequencies), coupled with \texttt{Spring} objects to enforce some harmonicity. Particles from distinct notes are coupled by \texttt{Gravity} objects, so that when polyphony is present, overtones align with one another. Needs a call to \texttt{tick()} to update the phases of its oscillators, and occasional (not necessarly sample-rate) calls to \texttt{physics()} to update the particles. 
  
  \item \texttt{Polyres<T>}: identical to \texttt{Polyphon<T>}, except that particle positions are used to set the center frequencies of a bank of resonant biquad filters. 
  
  \item \texttt{Granulator<T>}: granular synthesizer. Initialized with a pointer to a \texttt{Wave<T>}, the windowing function, and a pointer to a \texttt{Buffer<T>}, the buffer from which to read grains. The member function \texttt{request} attempts to allocate a new grain with some parameters (delay from current readhead of the buffer, grain length, playback speed, gain). Needs a call to \texttt{tick()} at audiorate. The \texttt{operator()} member function gets the current sample for output. 
  \end{enumerate} 

\end{document}